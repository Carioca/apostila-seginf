\chapter{Assinaturas Digitais}
\label{cha:assinaturas-digitais}

No capítulo anterior vimos que é possível construir um sistema de criptografia em que as partes não precisam trocar um segredo de antemão.
Os esquemas da criptografia assimétrica pressupõe a geração de um par de chaves, uma pública e outra secreta.
As assinaturas digitais, que veremos neste capítulo, seriam o equivalente àos códigos de autenticação de mensagem que vimos no Capítulo \ref{cha:mac} no contexto da criptografia assimétrica.

Um {\em sistema de assinatura digital} consiste de três algoritmos:
\begin{itemize}
\item $Gen$ recebe o parâmetro de segurança $1^n$ e produz um par de chaves $\langle s_k, p_k \rangle$. 
A primeira deve ser mantida em sigilo e a segunda pode ser publicada.
\item $Sign$ recebe a chave secreta $s_k$ e uma mensagem $m$ e produz uma assinatura $t$.
\item $Ver$ recebe a chave pública $p_k$, uma mensagem $m$ e uma assinatura $t$ e verifica se essa assinatura é válida para mensagem $m$ e de fato da pessoa que possui a chave secreta.
Em caso positivo o algritmo deve devoler $1$ e em caso negativo $0$. 
\end{itemize}

Note que no meio digital, para cada mensagem direferente uma mesma pessoa produz assinaturas diferentes.
Um sistema de assinatura digital é {\em correto} se:
\begin{displaymath}
  Ver(p_k, m, Sign(s_k, m)) = 1
\end{displaymath}
 
Um sistema como esse é {\em seguro contra falsificação} se qualquer adversário eficiente, mesmo com acesso a um oráculo que lhe entregue assinaturas para mensagens diferentes de $m$, não é capaz de produzir uma assinatura válida para $m$ com probabilidade considerável.

Da mesma forma que os sistemas de criptografia assimétrica, produzir uma assinatura digital para uma mensagem muito grande pode ser um processo lento.
Uma técnica usada na prática para mitigar este problema é assinar não a mensagem $m$ em si, mas um hash da mensagem $H(m)$.
Neste caso, para verificar a integridade e autenticidade da mensagem, basta gerar $H(m)$ e verificar a assinatura.
Essa construção é chamada de {\em paradigma Hash-and-Sign}.
É possível provar a segurança contra falsificação em uma construção como essa no caso em que o esquema de assinatura usado é seguro contra falsificação e o hash é resistente à colisão.

\section{Esquemas de Identificação}
\label{sec:esqu-de-ident}

Um esquema de identificação é um protocolo em que uma parte tem como objetivo provar sua identidade para a outra.
Chamamos de {\em provador} aquele que deseja provar sua identidade e {\em verificador} aquele que deseja verificá-la.
Assumimos que o verificador possui a chave pública do provador e vamos focar em protocolos com três interações e três algoritmos $langle \mathcal{P}_1, \mathcal{P}_2, \mathcal{V} \rangle$:
\begin{enumerate}
\item O provador usa $\mathcal{P}_1$ com sua chave secreta $s_k$ para gerar uma {\em mensagem inicial} $I$ e um estado $st$ e envia $I$ para o verificador.
\item O Verificador escolhe um {\em desafio} $r$ aleatoriamente de um conjunto $\Omega_{p_k}$ gerado a partida da chave pública do Provador e envia $r$  de volta.
\item O Provador usa $\mathcal{P}_2$ com entradas $s_k$, $st$ e $r$ para gerar uma resposta $s$ que ele envia para o Verificador.
\item Por fim, o Verificador testa se $\mathcal{V}(p_k, r, s)$ computa $I$.
\end{enumerate}

% DIAGRAMA!!!

A ideia por trás do esquema de identificação é que apenas que possui a chave secreta seria capaz de $s$ a partir do desafio $r$.
A mensagem $I$ e o estado $st$ servem para garantir que um adversário não copie os mesmos passos de identificação e seja bem sucedido.
Um esquema de identificação seguro deve garantir que seja computacionalmente inviável para um adversário enganar o sistema de identificação mesmo que ele observe vários processos similares.

A partir de um sistema de identificação seguro é possível gerar um sistema de assinatura digital seguro usando a {\em transformação de Fiat-Shamir} \cite{Fiat87}.
Na hora de assinar uma mensagem $I$ e $st$ são computados e calculamos $H(I, m)$ para gerar $r$ e usamos $r$, $s_k$ e $st$ para gerar $s$.
A assinatura será exatamente o par $\langle r, s \rangle$.
O algoritmo de verificação deve rodar $\mathcal{V}$ com $p_k$, $r$ e $s$ como entrada para produzir $I$ e então verifica-se se $H(I, m) = r$: 

\begin{itemize}
\item $Gen(1^n) := \langle s_k, p_k \rangle$ e assumimos que $H: \{0,1\}^* \to \Omega_{p_k}$ é uma função de hash específica
\item $Sign(s_k, m) = \langle r, s \rangle$ tal que
\begin{itemize}
\item $\mathcal{P}_1(s_k) := \langle I, st \rangle$,
\item $r := H(I, m)$ e
\item $s := \mathcal{P}_2(s_k, st, r)$
\end{itemize}
\item $Ver(p_k, m, \langle r, s \rangle) = \left\{
    \begin{array}{lcl}
      1 & \textrm{se} & H(\mathcal{V}(p_k, r, s), m) = r\\
      0 & \textrm{c.c.} &\\
    \end{array}
    \right.$
\end{itemize}

Considerando que o esquema de identificação $\langle \mathcal{P}_1, \mathcal{P}_2, \mathcal{V} \rangle$ seja seguro, podemos provar que o sistema de assinatura digital acima também é seguro, mas para esta prova é necessário supor não que $H$ seja resistente a colisões, mas que ele seja um {\em oráculo aleatório}, uma suposição muito mais forte e difícil de validar empiricamente.


\section{Assinatura RSA}
\label{sec:assinatura-rsa}

Um sistema de assinatura digital $\Pi = \langle Gen, Sign, Ver \rangle$ pode ser construído invertendo o esquema de criptografia assimétrica que vimos no capítulo anterior.
Ou seja, criptografamos a mensagem com a chave secreta para gerar a assinatura e descriptografamos com a chave pública para verificar.
Esse esquema não é seguro até aplicarmos o paradigma {\em hash-and-sign}.
O esquema todo segue:
\begin{itemize}
\item $Gen(1^n) := \langle s_k, p_k \rangle$ em que:
\begin{itemize}
\item $s_k := \langle N, e \rangle$ e
\item $p_k := \langle N, d \rangle$
\end{itemize}
\item $Sign(s_k, m) := [H(m)^d\ mod\ N]$
\item $Ver(p_k, m, t) := \left\{
    \begin{array}{lcl}
      1 & \textrm{se} & t^e \equiv H(m)\ mod\ N\\
      0 & \textrm{c.c.} &\\
    \end{array}
    \right.$
\end{itemize}

A correção deste sistema segue os mesmos passos da correção do sistema de criptografia RSA que vimos no capítulo anterior.
Além disso, podemos provar que se o sistema RSA é seguro e o hash $H$ usado é resistente à colisão então o sistema acima é seguro contra falsificação.
Esse sistema é a base do sistema de assinatura do protocolo {\tt RSA PKCS \#1 v2.1}.

\section{Algoritmo de Assinatura Digital (DSA)}
\label{sec:dsa}

O esquema de assinatura RSA depende da dificuldade do problema da fatoração.
Podemos também construir sistemas que dependem da dificuldade do problema do logaritmo discreto.
Este é o caso do popular esquema DSA e do ECDSA que usa curvas elípticas.
Antes de apresentar o sistema de assinatura digital, apresentaremos um esquema de identificação em que assumimos que a chave secreta é $x$ e a chave pública é $\langle \mathbb{G}, g, y, n \rangle$ com $\mathbb{G}$ um grupo cíclico, $g$ um gerador e $y = g^x$:

\begin{enumerate}
\item O Provador sorteia $k \leftarrow \mathbb{Z}_n^\star$ e envia $I := g^k$ para o Verificador.
\item O Verificador sorteia $a, r \leftarrow \mathbb{Z}_q$ como desafios e envia de volta.
\item O Provador envia $s := [k^{-1} \cdot (\alpha + xr)\ mod\ q]$ como resposta.
\item O Verificador aceita se $s \neq 0$ e $g^{\alpha s^{-1}} \cdot y^{rs^{-1}} = I$
\end{enumerate}

A probabilidade de $s = 0$ é desprezível.
Se assumirmos que $s \neq 0$ a correção do esquema segue, pois:
\begin{displaymath}
  g^{\alpha s^{-1}} \cdot y^{rs^{-1}} = g^{\alpha s^{-1}} \cdot g^{xrs^{-1}} = g^{(\alpha + xr)\cdot s^{-1}} = g^{(\alpha + xr)\cdot k \cdot (\alpha + xr)^{-1}} = g^k = I
\end{displaymath}

É possível provar que este esquema é seguro considerando que o problema do logaritmo discreto seja seguro.

Podemos transformar esse esquema de identificação em um sistema de assinatura usando a transformação de Fiat-Shamir, mas o popular esquema DSA segue um caminho um pouco diferente.
Assumiremos a existência de duas funções $H: \{0,1\}^* \to \mathbb{Z}_n$ e $F: \mathbb{G} \to \mathbb{Z}_n$:

\begin{itemize}
\item $Gen(1^n) = \langle s_k, p_k \rangle$ em que:
\begin{itemize}
\item $s_k := x \leftarrow \mathbb{Z}_n$ e
\item $p_k := \langle \mathbb{G}, g, y, n \rangle$ com $y = g^x$
\end{itemize}
\item $Sign(s_k, m) = \langle r, s \rangle$ em que:
  \begin{itemize}
  \item $r := F(g^k)$ com $k \leftarrow \mathbb{Z}_n$ e
  \item $s := [k^{-1} \cdot (H(m) + xr)\ mod\ q]$ (sorteamos outro $r$ caso $s = 0$)
  \end{itemize}
\item $Ver(p_k, m, t) := \left\{
    \begin{array}{lcl}
      1 & \textrm{se} & r = F(g^{H(m) \cdot s^{-1}} y^{r \cdot s^{-1}})\\
      0 & \textrm{c.c.} &\\
    \end{array}
    \right.$
\end{itemize}

A correção desse sistema é muito similar ao esquema de identificação apresentado anteriormente.
É possível provar sua segurança assumindo a dificuldade do problema do logaritmo discreto e tratando $H$ e $F$ como oráculos aleatórios.
No caso de $F$ isso é particularmente difícil de aceitar e não se conhece uma prova de segurança melhor do que essa até hoje.

\section{Infraestrutura de Chaves Públicas}
\label{sec:pki}

O protocolo de Diffie-Hellman e a criptografia assimétrica garantem a segurança da comunicação contra ataques passivos sem precisarmos supor que as partes compartilhe um segredo {\em a priori}.
Um problema que ainda não tratamos até aqui é como garantir a segurança contra um ataque ativo, ou seja, contra um modelo de ameaça em que o adversario não só é capaz de observar a comunicação, como também é capaz de interferir nela.
Um ataque que os sistemas que vimos até aqui estão particularmente sujeitos é o seguinte.
Digamos que Alice pretende se comunicar com Bob usando um sistema de criptografia assimétrico.
O primeiro passo para Alice é obter a chave pública de Bob.
Neste momento Eva pode enviar a sua chave pública como se fosse a de Bob, receber as mensagens que Alice enviaria para Bob, copiá-las e re-encaminhá-las para Bob.
Este tipo de ataque é chamado de {\em Man In The Middle} e nada do que vimos até aqui o previne.

Não há como evitar este tipo de ataque sem algum encontro físico em que algum segredo seja compartilhado de maneira segura, mas os sistemas de assinatura digitais podem facilitar bastante esse processo.
A ideia é confiar a alguma autoridade a identificação das chaves públicas.
A autoridade $A$ teria então a responsabilidade de verificar se a entidade portadora da identidade $Id_B$ é de fato a o pessoa que possui a chave secreta correspondente à chave pública $pk_B$.
Neste caso a autoridade $A$ pode emitir um {\em certificado} $cert_{A \to B}$ que nada mais é do que uma assinatura de $A$ conectando $Id_B$ a $pk_B$:
\begin{displaymath}
  cert_{A \to B} := Sign(Id_B, pk_B)
\end{displaymath}

Existem diversos modelos de {\em infraestrutura de chaves públicas} (PKI):
\begin{itemize}
\item {\em autoridade certificadora única:} Neste cenário assumimos que todos possuem a chave pública da autoridade certificadora $A$ que foi obtida de maneira segura. 
Sempre que algum novo ator $B$ precisar publicar sua chave pública ele deve se apresentar à essa autoridade e comprovar sua identidade para obter o certificado $cert_{A \to B}$
\item {\em múltiplas autoridades certificadoras:} Neste cenário existem várias autoridades certificadoras e assumimos que todos possuem chaves públicas de algumas ou todas elas.
Quando um novo ator precisar publicar sua chave ele pode procurar uma ou mais autoridades para gerar o certificado que será válido para aqueles que confiarem na autoridade certificadora que o emitiu.
Neste cenário a comunicação é tão segura quanto a menos confiável das autoridades incluídas na lista das partes.
\item {\em delegação de autoridade:} Neste cenário as autoridades certificadoras não apenas podem certificar a autenticidade de uma chave pública como podem também produzir um certificado especial $cert^*$ que dá o poder a outro de produzir certificados. 
Assim digamos que $C$ consiga um certificado de $B$ sobra a autenticidade de sua chave $cert_{B \to C}$ e suponha que $B$ possui um certificado especial de $A$ que o autorize a produzir certificados em seu nome $cert^*_{A \to B}$.
Assim alguém que possui a chave pública de $A$ pode verificar o certificado $cert^*_{A \to B}$ e usar a chave pública assinada de $B$ para verificar $cert_{B \to C}$.
\item {\em rede de confiança:} Neste cenário, todas as partes atuam como autoridades certificadoras e podem assinar certificados para qualquer chave que elas verificaram.
Cabe a cada um avaliar a confiança que tem em seus colegas quanto a edoniedade em produzir certificados.
\end{itemize}

Um certificado pode e deve conter uma {\em data de expiração} que limite seu tempo de validade.
Depois dessa data o certificado não dever ser mais considerado válido e precisa ser renovado.
Em alguns modelos o portador pode comprovar sua identidade com a autoridade que emitiu o certificado e esta pode assinar e publicar um {\em certificado de revogação} anunciando que a chave não deve mais ser considerada válida -- por exemplo no caso de ela ser perdida ou roubada.


\section{Exercícios}
\label{sec:exercicios}




