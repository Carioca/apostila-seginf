\chapter{Funções de Hash}
\label{cha:hash}

Uma função de hash mapeia uma sequência de bits de tamanho arbitrário em uma sequência curta e de tamanho fixo chamada {\em digest} ou {\em checksum}.
Em Estrutura de Dados estudamos funções de hash com o propósito de acessar uma lista em tempo $O(1)$.
Minimizar as colisões naquele contexto garantia que as listas ligadas de objetos associadas a cada índice de um vetor fosse a menor possível tornando a consulta mais eficiente.
Em aplicações de criptografia, evitar colisões é mais crítico, pois pode levar a vulnerabilidades no sistema.

Assim, uma função de hash é simplesmente uma função $H: \{0,1\}^* \to \{0,1\}^n$.
Para definir o conceito de resistência a colisão vamos introduzir artificialmete uma chave na função de hash que não precisa ser guardada em segredo.
Além disso, nosso modelo precisa conter além de $H$ um algoritmo $Gen$ que gera a chave.
O modelo, portanto, difere da construção prática.

Definiremos resistência à colisão para um sistema $\Pi = \langle Gen, H \rangle$ a partir do jogo que já nos abituamos:
\begin{enumerate}
\item O sistema usa $Gen$ que recebe $1^n$ e gera uma chave $s$.
\item O adversário $\mathcal{A}$ recebe $s$.
\item A devolve um par de mensagens $\langle x, x' \rangle$.
\end{enumerate}

O desafio de $\mathcal{A}$ é achar uma {\em colisão}, ou seja, um par $\langle x, x' \rangle$ tal que $H_s(x) = H_s(x')$
\begin{displaymath}
  HashCol_{\mathcal{A}, \Pi}(n) := \left\{
    \begin{array}{lcl}
      1 & \textrm{se} & H_s(x) = H_s(x')\\
      0 & \textrm{c.c.} &\\
    \end{array}
    \right.
\end{displaymath}

O sistema $\Pi$ é {\em resistente à colisão} se para todo adversário polinomial $\mathcal{A}$ existe uma função desprezível $\varepsilon$ tal que:
\begin{displaymath}
  Pr[HashCol_{\mathcal{A}, \Pi}(n) = 1] \leq \varepsilon(n)
\end{displaymath}

A necessidade de inserir uma chave é puramente técnica.
Sem uma chave não seria possível evitar que o adversário simplesmente pré-compute uma colisão e use-a para derrotar o jogo.
Na prática usamos funções sem chave e tratamos como resistentes a colisão quando isso for validade empiricamente.

Note que a resistência à colisão é uma propriedade mais forte do que outras propriedades desejáveis em funções de hash:
\begin{itemize}
\item {\em resistência contra colisões em alvos específicos}: dado $s$ e $x$ nenhum adversário eficiente é capaz de encontrar $x'$ tal que $H_s(x) = H_s(x')$ com probabilidade considerável.
\item {\em resistência contra preimagem}: dados $s$ e $y$ aleatório, nenhum adversário eficiente é capaz de encontrar $x$ tal que $h_s(x) = y$ com probabilidade considerável. 
\end{itemize}

Toda a função de hash está sujeita a ataques do tipo força bruta.
Ou seja, se $H: \{0,1\}^* \to \{0,1\}^n$ podemos calcular $H(x)$ para uma sequência de strings distintas $x_0, x_1, \dots, x_{2^n+1}$.
Pelo {\em princípio da casa dos pombos} necessariamente encontraremos neste caso uma colisão.
Na verdade se assumirmos que $H$ é uma função aleatória, podemos mostrar que para que a probabilidade de encontrar uma colisão seja maior do que $\frac{1}{2}$ precisamos nossa sequência strings deve ter cerca de $\Theta(\sqrt{n})$.
Esse resultado é chamado de {\em paradoxo do aniversário}, pois bastam 23 pessoas para garantir que a probabilidade de duas fazerem aniversário no mesmo dia seja maior que meio.
Assim, se quisermos um sistema que garanta a segurança equivalente a uma função aleatória com chave de 128 bits, precisamos usar uma função de hash muito confiável que produz uma saída com pelo menos 256 bits.

O ataque do aniversariante nos dá uma colisão qualquer $\langle x, x' \rangle$ a primeira vista isso pode parecer inofencivo, pois não podemos controlar os valores de $x$ e  $x'$.
Note, porém, que o ataque requer uma serie de pelo menos $\sqrt{n}$ mensagens, mas elas não precisam ser aleatórias.
Precisamos, portanto, apenas gerar um número suficente de mensagens equivalentes para que o ataque seja efetivo.
Essa é uma tarefa relativamente simples.
Considere o seguinte exemplo de um conjunto de mensagens equivalentes:


\begin{quote}
  É {\em difícil/impossível/desafiador/complicado} {\em imaginar/acreditar} que {\em encontraremos/localizaremos/contrataremos} outra {\em pessoa/empregada} com a mesma {\em capacidade/versatilidade/destreza} que Eva.
Ela fez um trabalho {\em incrível/excelente}.
\end{quote}

% mudar esse exemplo que está identico ao do livro

As palavras em itálico podem umas substituir as outras sem mudar significativamente o espírito da mensagem que pode ser escrita de 288 formas distintas.
Se nosso conjunto precisa ser $2^{32}$ basta escrever um texto com pelo menos 32 palavras em que cada uma possua um sinônimo.
 
\section{Construções}
\label{sec:construcoes}

A maioria das funções de hash seguem uma construção chamada {\em Merkle-Damgard} que assume a existância de uma {\em função de compressão} resistente à colisão para mensagens de tamanho fixo e a extende para mensagens de tamanho arbitrário.
Seja $\langle Gen, h \rangle$ um sistema de hash que comprime o tamanho de uma mensagem pela metade $h_s:\{0,1\}^{2n} \to \{0,1\}^n$.
Construimos $\langle Gen, H \rangle$ da seguinte maneira:
\begin{itemize}
\item Sejam $x \in \{0,1\}^*$, $|x| = L < 2^n$ e $B := \lceil \frac{L}{n} \rceil$ o número de blocos de $x$ de tamanho $n$ (se o tamanho de $x$ não for múltiplo de $n$ complete-o com um {\em pad} de $0$s) e insira $L$ ao fim de $x$ i.e. $x = x_0 \dots x_B L$.
\item Defina $z_0 := 0^n$.
\item Compute $z_i := h_s(z_{i-1}x_i)$ para $i = 1, \dots, B + 1$.
\item Devolva $z_{B+1}$.
\end{itemize}


\begin{theorem}
  Se $\langle Gen, h \rangle$ é resistente a colisão então $\langle Gen, H \rangle$ da forma como definido acima também é resistente a colisão.
\end{theorem}

Construir uma função de hash resistente a colisão para uma mensagem de tamanho arbitrário se resume, portanto, a encontrar uma para mensagem de tamanho fixo que a comprima pela metade.

\subsection{SHA-1}
\label{sec:sha-1}

O {\em Secure Hash Algorithm} (SHA-1) é um algoritmo da família MD4 que recebe uma entrada de tamanho arbitrário $l < 2^{64}$ e produz um {\em digest} de $160$ bits.
Antes de processar a mensagem é inserido um {\em pad} que consiste em uma sequência $10 \dots 0l$ em que $l = |x|$ é o número de bits da mensagem e $|x10 \dots 0l|$ é um múltiplo de 512.
Essa codificação de $x$ é então dividida em blocos $x_0, \dots, x_n$ tais que $|x_i| = 512$.

A função de hash $h$ que alimenta a construção de Merkle-Damgard consiste de 80 rodadas.
Um {\em message schedule} é responsavel por gerar 80 strings $W_0, \dots, W_{79}$ de 32 bits cada a partir do bloco $x_i$ sendo processado. 
Começamos com valores iniciais fixos $A =$ {\tt 67452301}, $B =$ {\tt EFCDAB89}, $C =$ {\tt 98BADCFE}, $D =$ {\tt 10325476}, $E =$ {\tt C3D2E1F0} e alteramos esses valores em cada rodada da seguinte maneira:

\begin{displaymath}
  A, B, C, D, E := (E + f_t(B,C,D) + A_{\lll 5} + W_j + K_t), A, B_{\lll 30}, C, D
\end{displaymath}

A cada 20 rodadas mudamos o valor da constante $K_t$ e da função $f_t$ de forma que o algoritmo precisa definir 4 constantes e quatro funções.
Apenas para matar a curiosidade seguem suas definições:
\begin{enumerate}
\item $K_t :=$ {\tt 5A827999} e $f_t(B,C,D) := (B \land C) \lor (B \land D)$ para $t = 0, \dots, 19$
\item $K_t :=$ {\tt 6ED9EBA1} e $f_t(B,C,D) := B \xor C \xor D$ para $t = 20, \dots, 39$ 
\item $K_t :=$ {\tt 8F1BBCDC} e $f_t(B,C,D) := (B \land C) \lor (B \land D) \lor (C \land D)$ para $t = 40, \dots, 59$
\item $K_t :=$ {\tt CA62C1D6} e $f_t(B,C,D) := B \xor C \xor D$ para $t = 60, \dots, 79$ 
\end{enumerate}

% falar da colisão encontrada no SHA-1 e no MD5

\section{Aplicações}
\label{sec:aplicacoes}

Funções de hash são amplamente utilizadas em protocolos de segurança.
Nesta seção veremos quatro aplicações bastante distintas desses sistemas: um sistema de autenticação bastante popular chamado HMAC, identificação de arquivos e outros tipos de mensagens simples e estruturadas  ({\em fingerprints} e {\em árvore de Merkle}) e derivação de subchaves a partir de outras chaves e a partir de uma senha.

\subsection{HMAC}
\label{sec:hmac}

No capítulo anterior vimos como construir um sistema de autenticação para mensagens de tamanho arbitrário aplicando um esquema similar ao modo CBC.
Alternativamente poderíamos utilizar um hash para produzir um {\em digest} da mensagem de tamanho fixo e então aplicar um sistema de autenticação para mensagens de tamanho fixo no resultado.
Esse sistema é conhecido como {\em Hash-and-MAC}.

Considere um sistema de autenticação $\Pi_M = \langle Gen_M, Mac_M, Ver_M \rangle$ para mensagens de tamanho fixo $l(n)$ e um sistema de hash $\langle Gen_H, H \rangle$ que produz um {\em digest} de tamanho $l(n)$.
O sistema {\em Hash-and-MAC} $\Pi = \langle Gen, Mac, Ver \rangle$ para mensagens de tamanho arbitrário é definido como:
\begin{itemize}
\item $Gen(1^n) := k = \langle k_M, s \rangle$ em que $Gen_M(1^n) := k_M$ e $Gen_H(1^n) := s$
\item $Mac(k, m) := Mac_M(k_M, H_s(m))$
\item $Ver(k, m, t) := \left\{
    \begin{array}{lcl}
      1 & \textrm{se} & Mac(k,m) = t\\
      0 & \textrm{c.c.} &\\
    \end{array}
    \right.$ 
\end{itemize}


\begin{theorem}
  Se $\Pi_M$ é um sistema de autenticação seguro contra falsificação e $\Pi_H$ é um sistema de hash resistente a colisão então a construção $\Pi$ acima é seguro contra falsificação. 
\end{theorem}

\begin{proof}
  Vamos apenas esboçar a prova e deixar os detalhes para o leitor.
  A ideia é a seguinte, suponha que o adversário $\mathcal{A}$ consulte o tag de uma série de mensagens $Q$ e então suponha por absurdo que ele consiga produzir um código válido para uma mensagem $m \notin Q$.
  Neste caso temos duas possibilidades:
  \begin{enumerate}
  \item existe uma mensagem $m' \in Q$ tal que $H_s(m') = H_s(m)$, mas isso contradiz a hipótese que $\Pi_H$ é resistente a colisões
  \item caso contrário $\mathcal{A}$ conseguiu falsificar um código para uma mensagem nova $H_s(m)$ de tamanho $l(n)$ contradizendo a hipótese que $\Pi_M$ é seguro contra falsificação.
  \end{enumerate}
\end{proof}

% Mencionar os protocolos que usam o HMAC

O popular sistema de autenticação HMAC se inspira no esquema apresentado acima usando duas vezes uma função de hash e duas constantes distintas {\tt ipad} e {\tt opad}.
Formalmente definimos o sistema HMAC da seguinte forma:
\begin{itemize}
\item $Gen(1^n) := k \leftarrow \{0,1\}^n$
\item $Mac(k, m) := H_s((k \xor \textrm{\tt opad}) H_s(k \xor \textrm{\tt ipad})m)$
\item $Ver(k, m, t) := \left\{
    \begin{array}{lcl}
      1 & \textrm{se} & Mac(k,m) = t\\
      0 & \textrm{c.c.} &\\
    \end{array}
    \right.$
\end{itemize}

% argumentar sobre a importância das duas constantes e a relação com o sistema anterior

\subsection{Fingerprints e Árvores de Merkle}
\label{label}

% fingerprint
% árvore de Merkle

\subsection{Funções de Derivação de Chaves}
\label{sec:kdf}

Uma {\em função de derivação de chaves} (KDF) tem como objeto produzir uma chave segura a partir de algum material contendo uma boa quantidade de entropia chamado {\em key material}.
Isso pode ser útil seja porque o {\em key material} não está suficientemente preparado para ser usado como uma chave -- por exemplo no caso de uma senha -- ou quando queremos derivar {\em subchaves} a partir de uma chave inicial.

Formalmente, uma função de derivação de chaves recebe com entrada o {\em key material} $\delta$, o tamanho $l$ da chave a ser produzida e opcionalmente um valor $r$ chamado de {\em salt} e um valor contextual $c$.
Um KDF seguro deve produzir a partir desses valores uma sequência de bits de tamanho $l$ {\em pseudoaleatória}.

Tipicamente um KDF consiste de duas etapas: 
\begin{enumerate}
\item {\em extração}: recebe o {\em key material} e produz uma sequência pseudoaleatória de bits de tamanho fixo
\item {\em expansão}: que expande o que foi produzido na fase de extração para uma sequência de bits de tamanho $l$
\end{enumerate}


% KDF e PBKDF

\section{Exercícios}
\label{sec:exercicios}


\begin{exercicio}
  Mostre que o seguinte sistema não é seguro contra falsificação considerando que o sistema de hash foi construído a partir do paradigma de Merkle-Damgard:
\begin{itemize}
\item $Gen(1^n) := k \leftarrow \{0,1\}^n$
\item $Mac(k,m) := H(km)$
\item $Ver(k,m,t) := \left\{
    \begin{array}{lcl}
      1 & \textrm{se} & Mac(k,m) = t\\
      0 & \textrm{c.c.} &\\
    \end{array}
    \right.$ 
\end{itemize}

\end{exercicio}

