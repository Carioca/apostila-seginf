\chapter{Funções de Mão Única}
\label{cha:owf}

Nos Capítulos \ref{cha:cifras-de-fluxo}, \ref{cha:cifras-de-bloco} e \ref{cha:mac} demonstramos que dadas certas suposições determinados sistemas de criptografia garantem a segurança contra determinados modelos de ataque.
Argumentamos que essas supoções -- a saber, a existência de geradores de números pseudoaleatórios, de funções pseudoaleatórias e de permutações pseudoaleatórias -- não são provadas, mas validadas empiricamente.
Neste capítulo argumentaremos porque não construimos sistemas que comprovadamente satisfazem essas suposições.
Para tanto apresentaremos uma condição necessária e suficiente para a existência de sistemas seguros: a existência de funções de mão única.
Por fim mostraremos que uma demonstração da existência de funções de mão única implicam que $P \neq NP$, ou seja, esse resultado teria como consequência um dos problemas em aberto mais relevantes na matemática hoje.

Uma {\em função de mão única} é uma função $f: \{0,1\}^n \to \{0,1\}^n$ que pode ser computada de maneira eficiente, mas que é inviável de ser invertida.
Ou seja, para todo adversário eficiente $\mathcal{A}$ que recebe $y := f(x)$ só consegue encontrar $x'$ tal que $f(x') = y$ com probabilidade desprezível.

% DEFINIÇÃO FORMAL

Note que para derrotar o desafio de uma função de mão única $\mathcal{A}$ precisa calcular a pré-imagem $x'$, não basta conhecer alguma informação parcial sobre $x'$.
Um {\em predicado hard-core} é um bit de informação sobre a $x'$ que é mantido escondido pela função de mão única $f$.
Formalmente, um predicado hard core é uma função $hc:\{0,1\}^n \to \{0,1\}$ se qualquer adversário eficiente que conhece $y := f(x)$ consegue acertar o valor de $hc(x)$ apenas com change desprezivelmente maior do que $\frac{1}{2}$.

% DEFINIÇÃO FORMAL

Seja $f:\{0,1\}^n \to \{0,1\}^n$ uma função de mão única.
Considere a função $g(x,r) := \langle f(x), r \rangle$ em que $|x| = |r|$ e defina $hc(x,r) := \bigoplus_{i=1}^nx_i \cdot r_i$.
É possível mostrar que $g$ é uma função de mão única e $hc$ é um predicado hard-core de $g$.
Esta construção é a base do seguinte teorema\footnote{Neste capítulo seguiremos essa abordagem de apresentar as construções e omitir as demonstrações.}:

\begin{theorem}[Teorema de Goldreich-Levin]
  Assuma que uma função de mão única existe. Então existe uma função de mão única com um predicado hard-core.
\end{theorem}

Uma vez extraído um bit ``seguro'' a partir de uma função de mão única, podemos usá-lo na construção de um PRG com fator de expanção $l(n) = n + 1$.

\begin{theorem}
  Seja $f$ uma função de mão única com um predicado hard-crore $hc$.
A função $G(s) := f(s)hc(s)$ é um PRG com fator de expansão $l(n) = n+1$.
\end{theorem}

Agora que somos capazes de gerar um PRG com fator de expansão mínimo, é possível expandi-lo para um PRG com fator de expansão $l(n) = p(n)$ em que $p$ é um polinomio qualquer.
Seja $G$ um PRG com fator de expansão mínimo, construimos $G'$ da seguinte maneira.
Usamos $G$ para gerar $l(n)+1$ bits, o último bit é usado como primeiro bit da sáida de $G'$ e os demais são usados como uma nova semente para $G$ e repetimos o processo quantas vezes forem nescessárias.
Esta construção é a base do seguinte teorema:


\begin{theorem}
  Se existe PRG com fator de expansão $l(n) = n + 1$ então para todo polinômio $p$ existe PRG $G'$ com fator de expansão $l(n) = p(n)$.
\end{theorem}

% DIAGRAMA DA CONSTRUÇÃO DE G'
  
É possível construir uma função pseudoaleatória a partir de um PRG com fator de expansão $2n$.
Seja $G(k) = y_0 \dots y_{2n}$ um PRG e defina $G_0(k) := y_0 \dots y_n$ e $G_1(k) := y_{n+1} \dots y_{2n}$.
Construiremos uma PRF $f_k: \{0,1\}^n \to \{0,1\}^n$ da seguinte maneira:
\begin{displaymath}
  f_k(x_1 \dots x_n) := G_{x_n}(\dots (G_{x_1}(k))\dots)
\end{displaymath}

\begin{theorem}
  Se $G$ é um PRG com fator de expansão $l(n) = 2n$ então a construção acima é um PRF.  
\end{theorem}

% DIAGRAMA DA CONSTRUÇÃO DE F

Para completar, a partir de uma PRF é possível construir uma PRP usando uma rede de Feistel com três passos em que cada passo usa uma chave distinta.
\begin{displaymath}
  f_{k_1,k_2,k_3}^{(3)}(x) := Feistel_{f_{k_1},f_{k_2},f_{k_3}}(x)
\end{displaymath}
Lembrando que:

\begin{eqnarray*}
  Feistel_{f_1,\dots,f_n}(x) & := & L_nR_n\\
  L_i & := & R_{i-1}\\
  R_i & := & L_{i-1} \xor f_i(R_{i-1})
\end{eqnarray*}


\begin{theorem}
  Se $f$ é uma PRF então $f^{(3)}$ é uma permutação pseudoaleatória.
\end{theorem}

% DIAGRAMA DA CONSTRUÇÃO DE F3

Resumindo, a existência de uma função de mão única é uma condição suficiente para a existência de PRG, PRF e PRP.
No Capítulos \ref{cha:cifras-de-fluxo} vimos que a existência de um PRG é condição suficiente para construir um sistema seguro contra ataques {\em ciphertext only} e nos Capítulos \ref{cha:cifras-de-bloco} e \ref{cha:mac} vimos que a existência de uma PRF é condição suficiente para construção de sistemas de criptografia seguros contra CPA e contra CCA, bem como para construção de sistemas autenticados.

O teorema a seguir mostra a condição reversa, ou seja, que a existência de função de mão única é condição necessária para existência de todas essas construções:


\begin{theorem}
  Se existe um sistema seguro contra ataques {\em ciphertext only} que protega uma mensagem duas vezes maior do que sua chave então existe uma função de mão única.
\end{theorem}

\begin{proof}
  % Esta pretendo fazer passo a passo
\end{proof}

% Diagrama
  % owf => PRG => PRF => PRP => Segurança => owf
% Função de Mão Única => P != NP