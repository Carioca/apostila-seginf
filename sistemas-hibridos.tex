\chapter{Sistemas Híbridos}
\label{cha:sistemas-hibridos}

Os sistemas de criptografia assimétricos que vimos no capítulo anterior são pelo menos uma ordem de grandesa menos eficientes do que os sistemas de criptografia simétrica que vimos anteriormente.
Além disso, é difícil garantir a segurança desses sistemas pra mensagens em qualquer distribuição de probabilidades.
Por esses motivos, o modelos mais popular de criptografia assimétrica segue um modelo híbrido dividido em duas fases: uma fase de encapsulamento de chave (KEM) seguido de uma fase de encapsulamento dos dados (DEM).

Um {\em mencanismo de encapsulamento de chave} (KEM) é um sistema $\Pi = \langle Gen, Encaps, Decaps \rangle$ tal que:
\begin{itemize}
\item $Gen(1^n) := \langle sk, pk \rangle$ em que $pk$ é uma chave pública e $sk$ uma chave secreta
\item $Encaps$ recebe a chave pública $pk$ e o parâmetro de segurânça $1^n$ e produz uma cifra $c$ e uma chave $k \in \{0,1\}^{l(n)}$
\item $Decaps$ recebe a chave secreta $sk$ e a cifra $c$ e produz uma chave.
\end{itemize}
 
Um sistema KEM em que $Encaps(pk, 1^n) = \langle c, k \rangle$ é {\em correto} se:
\begin{displaymath}
  Decaps(sk, c) =  k
\end{displaymath}

Um sistema KEM é {\em seguro contra CPA} se um adversário polinomial não é capaz de distinguir com probabilidade considerável a chave produzida por $Encaps$ de uma chave de mesmo tamanho escolhida aleatóriamente.
Se $\Pi_K = \langle Gen_K, Encaps, Decaps \rangle$ é um KEM seguro contra CPA, e $\Pi' = \langle Gen', E', D' \rangle$ é um sistema de criptografia simétrica seguro contra ataques ``ciphertext only'' então o seguinte sistema $\Pi = \langle Gen, E, D \rangle$, chamado de {\em híbrido}, é seguro contra CPA:

\begin{itemize}
\item $Gen(1^n) := Gen_k(1^n) = \langle sk, pk \rangle$
\item $E(pk, m) := \langle c_k, c \rangle$ em que: 
\begin{itemize}
\item $Encaps(pk, 1^n) = \langle c_k, k \rangle$ e
\item $E'(k, m) = c$
\end{itemize}
\item $D(sk,\langle c_k, c \rangle) = m$ tal que:
\begin{itemize}
\item $Decaps(sk, c_k) = k$ e
\item $D'(k, c) = m$
\end{itemize}
\end{itemize}

Uma forma natural de produzir um KEM é produzir uma chave $k$ e usar um sistema de criptografia assimétrica $\Pi$ para criptografar a chave.
Esse esquema é seguro desde que o sistema desde que $\Pi$ seja seguro, mas veremos que existem sistemas mais eficientes em determinados casos.

\section{El Gammal}
\label{sec:el-gammal}

O principal inconveniente do sistema de El Gammal é que $M = \mathbb{G}$.
Ou seja, as mensagens devem ser elementos do grupo e não sequências arbitrárias de bits.
Não se deve restringir que mensagens os usuários devem ser capazes de trocar por conta de uma questão técnica como essa.
Uma forma de resolver esse problema é utilizando um sistema híbrido.

Uma forma de obter um sistema híbrido é simplemente usando o sistema de El Gamal para criptografar uma chave.
Ao invés disso, porém, podemos construir o seguinte KEM:

\begin{itemize}
\item $Gen(1^n) := \langle sk, pk \rangle$ em que e $\mathcal{G}(1^n) := \langle \mathbb{G}, g \rangle$, 
\begin{itemize}
\item $sk = \langle \mathbb{G}, g, x \rangle$ com $x \leftarrow \mathbb{Z}_n$ e
\item $pk = \langle \mathbb{G}, g, h, H \rangle$ com $H: \mathbb{G} \to \{0,1\}^{l(n)}$ e $h = g^x$
\end{itemize}
\item $Encaps(pk, 1^n) = \langle g^y, H(h^y)\rangle$ em que $y \leftarrow \mathbb{Z}_n$
\item $Decaps(sk, c) = H(c^x)$
\end{itemize}

A função $H$ é simplesmente uma função de hash.
Temos assim que $Decaps(sk, g^y) = H((g^y)^x) = H(g^{xy}) = k$, pois $k = H(h^y) = H(g^{xy})$.
É possível provar que este mecanismo de encapsulamento de chaves é tão seguro quanto o protocolo de Diffie-Helmann.
Usamos então $k$ como chave para criptografar uma mensagem com um sistema de criptografia simétrica para produzir um sistema híbrido.
Como já enunciamos, se o sistema de criptografia simétrico usado for seguro contra ataques ``ciphertext only'' então o sistema híbrido será seguro sob CPA.

\section{RSA}
\label{sec:rsa}

Como no caso da cifra de El Gamal, podemos construir um Mecanismo de Encapsulamento a partir do RSA.
A construção segue os seguintes passos:
\begin{itemize}
\item $Gen(1^n) := \langle sk, pk \rangle$ em que:
\begin{itemize}
\item $pk = \langle N, e \rangle$
\item $sk = \langle N, d' \rangle$ em que $d' = [d^n\ mod\ \phi(N)]$
\end{itemize}
\item $Encaps(pk, 1^n) = \langle c_{n+1}, k \rangle$ em que $c_1 \leftarrow \mathbb{Z}_N^\star$:
\begin{itemize}
\item $k_i := \textrm{\tt lsb}(c_i)$ e
\item $c_{i+1} := [c_i^e\ mod\ N]$
\end{itemize}
\item $Decaps(sk, c) = k'$ em que $c_1' = [c^{d'}\ mod\ N]$ e:
\begin{itemize}
\item $k_i := \textrm{\tt lsb}(c_i)$ e
\item $c_{i+1} := [c_i^e\ mod\ N]$
\end{itemize}
\end{itemize}

Em palavras, a ideia é criptografar o primeiro bit de $k$ usando o último bit de $[c^e\ mod\ N]$ para um $c$ escolhido aleatóriamente em $\mathbb{Z}_N^\star$.
O segundo bit de $k$ é computado usando o último de $[c^{e^2}\ mod\ N]$ e assim por diante.
Ao final o algoritmo $Encaps$ devole o $k$ obtido e $[c^{e^n}\ mod\ N]$.
Para decifrar partimos de $[c^{e^n}\ mod\ N]$ e elevamos a $d^n$ para recuperar o $c$ original e então repetimos exatamente o mesmo processo do algoritmo $Encaps$ para extrair cada um dos bits de $k$.

É possível mostrar que esta construção produz um KEM seguro e, portanto, pode ser usado para produzir um sistema híbrido seguro contra CPA como vimos.
Construir um sistema RSA seguro contra CCA requer uma série de modificações nos esquemas que vimos até aqui, cujos detalhes omitiremos.
A especificação de um sistema RSA seguro contra CCA foi formalizada em um sistema chamado {\tt RSA PKCS \#1 v2.0}.