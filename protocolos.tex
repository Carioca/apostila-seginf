\chapter{Protocolos}
\label{cha:protocolos}

Para fechar essas notas apresentaremos uma serie de protocolos que utilizam as primitivas criptograficas que vimos até aqui.
Um {\em protocolo} propriamente dito deveria descrever os passos de comunicação em um nível de detalhes suficiente para ser implementado sem ambiguidades.
Este não é nosso propósito.
Pretendemos aqui apenas apresentar aplicações práticas de uso de criptografia forte e como as primitivas são usadas para resolver problemas de comunicação digital.

\section{Transport Layer Security}
\label{sec:tls}

O {\em Transport Layer Security} (TLS) é a evolução do protobolo {\em Secure Socket Layer} (SSL) usado amplamente em todo o mundo nas conexões web suguras (sempre que conectamos em um site {\tt https}).
A primeira versão do SSL data de meados dos anos 90 e foi desenvolvida por programadores da Netscape.
A primeira versão do TLS é de 1999 e sua versão atual é de 2006.

O protocolo se divide em duas fases: {\em handshake} responsável por estabelecer as chaves de critpografia simétrica entre cliente e servidor e {\em record-layer} responsavel por garantir a confidencialidade, integridade e autenticidade na comunicação. 
Na comunicação usando TLS o cliente possui a chave pública de uma série de autoridades certificadoras que tipicamente vem junto do navegador e o servidor deve possuir um certificado $cert_{CA \to S}$ que obteve junto a alguma dessas entidades certificadoras para um par de chaves $\langle sk_S, pk_S \rangle$.
O {\em handshake} consiste dos seguintes passos:

\begin{enumerate}
\item O Cliente $C$ envia uma mensagem a $S$ contendo informando a versão do protocolo e os esquemas de criptografia suportados e um {\em nonce} $N_C$
\item $S$ seleciona a última versão do protocolo compatível com $C$ e envia de volta sua chave pública $pk_S$ com o certificado $cert_{CA \to S}$ e outro nonce $N_S$
\item $C$ verifica o certificado $cert_{CA \to S}$ então usa $pk_S$ para encapsular uma chave ({\em pre-master key}) $Encaps(pk_S, 1^k) = \langle c_{pmk}, pmk \rangle$.
  Por meio de uma função de derivação de chave $KDF$ é então produzida outra chave $mk$ ({\em master key}) usando $pmk$, $N_C$ e $N_S$.
Um PRG é usado para gerar quatro chaves $k_C$, $k_C'$, $k_S$ e $k_S'$.
Por fim, $C$ produz um mac $t_C$ com a chave $mk$ de toda a comunicação entre as partes até este ponto e envia $t_C$ e $c_{pmk}$ para $S$.
\item $S$ computa $Decaps(sk_S, c_{pmk})$ para gerar $pmk$ e então segue os mesmos passos de $C$ para recuperar $k_C$, $k_C'$, $k_S$ e $k_S'$.
Em seguida $S$ verifica $t_C$ e aborta o processo em caso de falha.
Por fim, $S$ produz outro mac $t_S$ de toda a comunicação até aqui com a chave $mk$ e envia para $C$.
\item $C$ verifica $t_S$ e aborta o processo se falhar.
\end{enumerate}

%% DIAGRAMA ?

No fim deste processo $S$ e $C$ compartilham dois pares de chaves autênticos.
Quando $C$ envia uma mensagem para $S$ ele a autentica com $k_C'$ e a criptografa com $k_C$ e quando $S$ responde ele autentica a mensagem com $k_S'$ e criptografa tudo com $k_S$.
Note que o protocolo segue o paradigma {\em mac then encrypt} que não é demononstradamente seguro.


\section{Secure Shell}
\label{sec:ssh}

O protocolo {\em Secure Shell} (SSH) para comunicação segura para login e transferência de arquivos em servidores remotos.

Antes da fase de troca de chaves, as partes passam por uma fase de negociação de algoritmos.
Nesta fase cada qual envia para a outra parte uma lista dos algortimos de criptografia suportados $I_S$ e $I_C$.
Nesta fase eles definem um grupo $\mathbb{G}$ com ordem $n$ e gerador $g$ e uma função de hash $H$.
A fase de troca de chaves segue os seguintes passos:
\begin{enumerate}
\item $C$ sorteia $x \leftarrow \{2, \dots, n-1\}$ e envia $e = g^x$ para $S$.
\item $S$ sorteia $y \leftarrow \{2, \dots, n-1\}$, calcula $f = g^y$ e computa $k = e^y$, $h = H(Id_C || Id_S || I_C || I_S || pk_S || e || f || k)$ e $s = Sign(sk_S, H)$.
$S$ então envia $pk_S$, $f$ e $s$ para $C$
\item $C$ pode verificar a autenticidade de $pk_S$ neste passo, então computa $k = f^x$ e $h = H(Id_C || Id_S || I_C || I_S || pk_S || e || f || k)$ e, por, fim verifica a assinatura $Ver(pk_S, h, s)$.
\end{enumerate}

O protocolo SSH permite o uso de certificados para verificar $pk_S$.
Tipicamente, porém, o modelo de segurança que é seguido é chamado de {\em Trust On First Use} (TOFU): a primeira vez que o cliente $C$ se conecta com o servidor $S$ uma aviso é exibido e a identidade do servidor é salva em uma base de dados no cliente, assim as próximas conexões podem verificar a autenticidade da chave apenas verificando essa base.
A conexão apresenta também um {\em fingreprint} da chave pública que pode ser enviado de maneira segura para verificação manual.

No final da fase de troca de chaves servidor e cliente compartilham $k$, $h$ e uma identificação da seção $Id$.
A partir desses valores são geradas as chaves de criptografia $k_S$ e $k_C$, as chaves de autenticação $k_S'$ e $k_C'$ e os vetores inciais $IV_S$ e $IV_C$.
Todos valores são obtidos usando um hash $H$ com $k$, $h$, $Id$ e uma constante diferente para cada chave.

Antes das partes começarem a fase de troca de mensagens {\em protocolo de conexão}, o servidor precisa autenticar o cliente.
Isso é feito na fase de {\em autenticação} de três formas alternativas:
\begin{itemize}
\item {\em chave pública}: $C$ envia uma mensagem assinada para $S$ contendo $pk_C$. 
O servidor então verifica se a chave é aceitável para a autenticação e então verifica a assinatura.
\item {\em senha}: $C$ envia uma mensagem com uma senha protegida pelo esquema apresentado acima.
\item {\em hostbased}: O host de $C$ realiza a autenticação por meio de algum protocolo de verificação de assinatura.
O servidor verifica o host e confia que ele autenticou $C$.
\end{itemize}

\section{Pretty Good Privacy}
\label{sec:pgp}

O PGP foi desenvolvido no começo dos anos 90 por Phil Zimmermann, um ativista do movimento contra o uso de armas nucleares.
O protocolo utiliza criptografia assimétrica e o modelo de rede de confiança para certificação das chaves públicas.
Assim cada usuário possui um par de chaves $\langle sk, pk \rangle$ e disponibiliza $pk$ em um servidor de chaves.
Qualquer um pode verificar a autenticidade de uma chave -- conferindo seu fingerprint manualmente e a identidade do titula -- e publicar no mesmo servidor de chaves um certificado.
Apesar da idade, o PGP é ainda hoje o principal protocolo de criptografia ponta a ponta para e-mails.

Com protocolo PGP um usuário pode: criptografar uma mensagem, assinar uma mensagem ou assinar e criptografar a mensagem.
\begin{itemize}
\item {\em criptografia:} Alice sorteia uma chave de seção $k$, compacta a mensagem $m$ e criptografa o resultado $Z(m)$ com uma cifra de bloco em modo {\em Cipher Feedback} (CFB) $c = E(k, Z(m))$ então a chave $k$ é criptografada usando uma versão do algoritmo RSA ou de El Gamal $c_k = E'(pk_B, k)$ e ambos $c$ e $c_k$ são enviados para Bob.
  Bob então usa sua chave secreta $sk_B$ para recuperar $k$ e então usa $k$ para desctiptografar e por fim então descompatar $m$.
  Em poucas palavras, trata-se de um KEM tradicional com um passo de compactação.
\item {\em autenticação}: Alice computa $H(m)$ e assina com sua chave secreta $sk_A$ usando o algoritmo RSA ou DSA $Sign(sk_A, H(m))$, junta isso com a mensagem $m$ e compacta tudo $Z(m||Sign(sk_A, H(m)))$ para mandar para Bob.
Bob então decompacta a mensagem, calcula o hash e verifica a assinatura.
Ou seja, trata-se de uma implementação simples do modelo {\em hash-and-sign}.
\item {\em autenticação e criptografia}: Alice autentica a mensagem e em seguida criptografa $E(k, Z(m||Sign(sk_A, H(m))))$ e Bob faz os passos na ordem reversa. 
\end{itemize}

O modo {\em cipher feedback} (CFB) se assemelha bastante ao modo CBC que vimos anteriormente:

\begin{eqnarray*}
  c_0 & := & IV \leftarrow \{0,1\}^n\\
  c_i & := & E(k, c_{i-1}) \xor m_i
\end{eqnarray*}


\section{Off The Record}
\label{sec:otr}

Enquanto o protocolo PGP foi concebido para comunicação assíncrona, estilo e-mail, modelando essencialmente o que seria o envio de uma carta selada, o protocolo OTR procura modelar uma conversa privada em uma comunicação síncrona, como um chat \cite{Borisov04}.
Como o PGP, o OTR busca garantir autenticidade, integridade e confidencialidade na comunicação, mas além disso o protocolo tenta também garantir:
\begin{itemize}
\item {\em perfect foward secrecy}: caso alguém grave a comunicação cifrada entre as partes e eventualmente tenha acesso a chave de comunicação não será possível decifrar as mensagens antigas
\item {\em negação plausível}: o destinatário da mensagem não deve ser capaz de provar a um terceiro quem foi o remetente da mensagem
\end{itemize}

Como outros protocolos que vimos, o OTR começa com um handshake.
\begin{itemize}
\item Alice gera $x_0$, produz $g^{x_0}$, assina com sua chave secreta e envia $Sig(sk_A, g^{x_0})$ e $g^{x_0}$ para Bob
\item Bob então verifica a assinatura, gera $y_0$, produz $g^{y_0}$ e envia $Sign(sk_B, g^{y_0})$ e $g^{y_0}$ para Alice
\end{itemize}

Assumimos que Alice e Bob possuem cada qual a chave pública do outro.
Eles podem verificar manualmente o fingerprint dessa chave ou usar algum outro protocolo para isso\footnote{No caso da comunicação síncrona é possível usar, por exemplo, o protocolo do milionário socialista para verificar se as partes compartilham alguma informação secreta}.
Uma vez que as partes foram autenticadas, elas podem usar $k_{00} = H(g^{x_0y_0})$ como chave efêmera para encriptar as mensagens.
O esquema de criptografia usado no OTR é o AES em modeo contador que é propositalmente maleável -- ou seja, é fácil injetar uma mensagem.
A comuncação então segue os seguintes passos:
\begin{itemize}
\item Alice gera $x_1$, calcula $E(k_{00}, m_1) = c_1$ e $MAC(H(k_{00}), c_1 || g^{x_1}) = t_1 $ e envia $c_1$, $t_1$ e $g^{x_1}$ para Bob.
\item Bob gera $y_1$ e calcula $k_{10} = H(g^{x_1y_0})$ e $g^{y_1}$ e então manda $E(k_{10}, m_2) = c_2$, $g^{y_1}$, $MAC(H(k_{10}), c_2 || g^{y_1}) = t_2$ e $H(k_{00})$ para Alice.
O ponto importante aqui é que $t_1$ já cumpriu seu papel e então sua chave pode ser transmitida pelo canal inseguro. 
\item Alice então pode esquecer (apagar) $x_1$ e gerar um novo $x_2$, caluclar $g^{x_2}$ e repitir todo o processo.
\end{itemize}

O {\em perfect foward secrecy} é garantido no processo de troca e esquecimento de chaves ({\em rechaveamento}).
Já a {\em negação plausível} é ao se revelar a chave de do MAC e no uso de um esquema de criptografia maleável.


\section{Signal}
\label{sec:signal}

O OTR foi concebido como um protocolo para criptografia ponta a ponta para comunicação síncrona.
Os mensageiros modernos, porém, operam tipicamente de maneira assíncrona -- ou seja o usuário não precisa saber se a outra parte está ou não conectada na hora em que a mensagem é enviada.
Essa característica traz alguns desafios particulares em termos de segurança.
O protocolo do Signal (originalmente chamado de Axolotl) soluciona alguns desses problemas.

Como seus antecessores, ele opera em duas fases: handshake e troca de mensagens.
Quando o aplicativo é instalado, Alice gera uma chave permanente $A$ que fica armazenada no servidor responsável por distribui-la para seus contatos (no nosso exemplo Bob).
Diferente do OTR, o handshake do Signal não usa assinaturas digitais e segue os seguintes passos.:
\begin{itemize}
\item Alice sorteia uma chave efêmera $a_0$ e envia $g^{a_0}$ para Bob.
\item Bob sorteia uma chave efêmera $b_b$ e envia $g^{b_0}$ para Alice.
\item Amabas as partes produzem três valores combinando suas chaves permanentes e efêmeras $g^{Ab_0}$, $g^{Ba_0}$ e $g^{a_0b_0}$.
  Esses valores alimentam um HKDF que gera uma chave $rk_0$ chamada de chave raiz ({\em root key}).
\end{itemize}

Essa forma de handshake dispensa a implementação delicada de assinaturas digitais e fortalece a negação plausível.
Alice pode garantir a autenticidade pois apenas Bob poderia gerar $g^{Ba_0}$, mas ela não tem como provar isso pois $a_0$ é um valor aleatório que não tem qualquer conexão com sua identidade.

No OTR o rechaveamento ocorre sempre que uma comunicação se completa, ou seja, Alice escreve para Bob e Bob responde para Alice.
Caso Alice escreva múltiplas mensagens para Bob sem resposta, a mesma chave é usada.
A solução {\em ad hoc} que o OTR oferece para essa limitação é forçar que Bob responda uma mensagem vazia depois de um número máximo de mensagens recebidas.
O Signal resolve esse problema de uma forma mais elegante.

A chave raiz alimenta o KDF para gerar uma {\em chain key} $ck$ que por sua vez alimenta um MAC para gerar uma chave mestra ({\em master key}) $mk$.
A {\em master key}, por fim, alimenta o KDF e gera as chaves para criptografar $k$ e autenticar $k'$ a mensagem a ser enviada.
Quando um ciclo de comunicação se fecha, a rechaveamos a raiz (exatamente como no OTR), mas quando a mesma parte envia mais de uma mensagem geramos uma nova {\em chain key} e esquecemos a anterior.
As chaves mestras são armazenadas até que sua mensagem correspondente chegue (por isso as mensagem precisam ser numeradas), mas note que a partir delas não é possível derivar as chaves das próximas mensagens ou das mensagens anteriores.
Esse esquema chama-se {\em ratchet}.

Os servidores do Signal armazenam um número de {\em pré-chaves} $g^a$ para cada usuário para o caso em que um handshake ocorra quando eles estão offline.
Para garantir a autenticidade da chave permanente o protocolo segue o modelo do SSH.
Ou seja, é possível verificar manualmente o {\em fingerprint} da chave, mas principalmente o que ocorre é a confiança no primeiro acesso (TOFU) e um aviso com destaque caso a chave mude.
Por fim, as mensagens armazenadas no aparelho são criptografadas localmente com uma cifra de bloco AES e autenticação HMAC cujas chaves são geradas aleatóriamente e armazenadas também de maneira criptografada usando o PBKDF2.

O protocolo do Signal foi originalmente implementado em um aplicativo open source de troca de SMS chamado Textsecure.
Em XXX o Textsecure incorporou a capacidade de envio de mensagens via internet e em XXX mudou de nome para Signal.
Em XXX os desenvolvedores do Signal implementaram esse protocolo no Whatsapp.
Além de ser um aplicativo de código aberto, o Signal se diferencia do Whatsapp por sua política de privacidade.
Enquanto o aplicativo do Facebook se reserva o direito de usar os metadados da comunicação comercialmente, o Signal armazena o mínimo desses dados -- apenas a data e hora da instalação e a data e hora da última conexão.

%\section{Telegram}
%\label{sec:telegram}

%\section{Blockchain}
%\label{sec:blockchain}


